\documentclass[a4paper,12pt]{article}
\usepackage[a4paper, total={180mm, 272mm}]{geometry}

\usepackage{fontspec}
\setmainfont[Path=fonts/, Extension=.ttf]{ipaexm}

\setlength\parindent{3.5em}
\setlength\parskip{0em}
\renewcommand{\baselinestretch}{1.247}

\begin{document}

\thispagestyle{empty}

\Large
\noindent \\
Negate Ino\medskip
\par
\normalsize
絵の色チャンネルごとに反転します。\\
\par
\ \ 8bits 画像: 0-{-}> \ \ \ 255、1-{-}> \ \ \, 254、... \ \ \ \ 254-{-}>1、 \ \ \, 255-{-}>0\par
16bits 画像: 0-{-}>65535、1-{-}>65534、... 65534-{-}>1、65535-{-}>0\par
という様に値が反転します。\\
\par
Alpha値があるとき、その Alpha値に対して反転します。\par
Alphaが \ \ \ 0: 0-{-}> \ \ \, 0、1-{-}> \ \ \ 0、... 254-{-}>0、255-{-}>0\par
Alphaが128: 0-{-}>128、1-{-}>127、... 127-{-}>1、128-{-}>0、...255-{-}>0\par
Alphaが255: 0-{-}>255、1-{-}>254、... 254-{-}>1、255-{-}>0\par
となり、これによりセル画像は正常に反転します。\\
\\
-{-}- \ 入力 \ -{-}-\\
Source\par
処理をする画像を接続します。\\
\\
-{-}- \ 設定 \ -{-}-\\
各色チャンネル毎に反転するかどうか指定します。\\
\\
Red \ \ \ \, 赤チャンネル反転スイッチ\\
Green \ 緑チャンネル反転スイッチ\\
Blue \ \ \, 青チャンネル反転スイッチ\\
Alpha \ \ Alphaチャンネル反転スイッチ\\
\\
OFF ではそのチャンネルには何もしません。\\
ON で、対象チャンネルを反転します。\\
初期値は、Red,Green,Blue が ON、Alphaが OFF、です。

\end{document}