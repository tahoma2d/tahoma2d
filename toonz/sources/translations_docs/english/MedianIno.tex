\documentclass[a4paper,12pt]{article}
\usepackage[a4paper, total={180mm, 272mm}]{geometry}

\usepackage{fontspec}
\setmainfont[Path=fonts/, Extension=.ttf]{ipaexm}

\setlength\parindent{3.5em}
\setlength\parskip{0em}
\renewcommand{\baselinestretch}{1.247}

\begin{document}

\thispagestyle{empty}

\Large
\noindent \\
Median Ino\medskip
\par
\normalsize
Reduces noise, and erode majority of middle colors, rounds the contour of\par 
the picture\\
\\
-{-}- \ Inputs \ -{-}-\\
Source\par
Connect the image to be processed.\\
\\
Reference\par
Connect the reference image to assign the strength of the effect into each pixel.\\
\\
-{-}- \ Settings \ -{-}-\\
Radius\par
Specify the area to be eroded by a circle radius.\par
The unit is mm.\\
\par
Specify a value greater than or equal to 0. The maximum is 100mm.\par
A value smaller than a pixel width (because it does not include any surrounding\par 
pixels) will do nothing.\\
\par
The default value is 0.35mm.\\
\\
Channel\par
Specifies the image color channel to apply the median to.\\
\par
\textquotedbl Red\textquotedbl\par
\textquotedbl Green\textquotedbl\par
\textquotedbl Blue\textquotedbl\par
\textquotedbl Alpha\textquotedbl\par
Select to process the specified color channel,\par
it will store the results in the RGBA channels.\par
In black-and-white images, using this method of single-channel processing,\par
the speed of processing will be faster.\\
\par
\textquotedbl All\textquotedbl\par
This will process all RGBA channels.\\
\par
The default setting is \textquotedbl All\textquotedbl .\\
\newpage

\thispagestyle{empty}

\ \vspace{-0.2em}
\\
Reference\par
Choose how the Reference image values are used to set the strength of the effect\par 
into each pixel.\par
Choose from Red/Green/Blue/Alpha/Luminance.\par
Choose Nothing to disable the effect.\par
The default value is Red.

\end{document}