\documentclass[a4paper,12pt]{article}
\usepackage[a4paper, total={180mm, 272mm}]{geometry}

\usepackage{fontspec}
\setmainfont[Path=fonts/, Extension=.ttf]{ipaexm}

\setlength\parindent{3.5em}
\setlength\parskip{0em}
\renewcommand{\baselinestretch}{1.247}

\begin{document}

\thispagestyle{empty}

\Large
\noindent \\
Fog Ino\medskip
\par
\normalsize
It scatters light.\par
Allows for the creation of light scattering effects in water, bath, fog, etc.,\par
It aims at creating effects like fog and diffusion in optical filters, but it does\par 
not actually simulate light.\\
\par
Each pixel, is affected by the shining of brighter nearby pixels.\par
Effect from closer pixels will be stronger, while it will have a weaker impact from\par 
distant pixels.\\
\par
When \textquotedbl Alpha Rendering\textquotedbl \ is active, Alpha channel will be processed first,\par
then it will process all RGB pixels where the Alpha channel is not zero.\\
\par
When \textquotedbl Alpha Rendering\textquotedbl \ is inactive, Alpha channel will not be taken into account,\par
so the RGB image changes will not be masked, causing jagged edges.\\
\\
-{-}- \ Inputs \ -{-}-\\
Source\par
Connect the image to be processed.\\
\\
-{-}- \ Settings \ -{-}-\\
Radius\par
The extent to which scatter light, specified as a circle radius.\par
The unit is millimeters.\par
Specify a value greater than or equal to 0. The maximum is 100.\par
When the value is smaller than a pixel, no light will be scattered and fog will not\par 
be applied.\par
The larger the Radius, the longer it will take to process.\par
The default value is 1.\\
\\
Curve\par
The attenuation curve of scattered light.\par
Specify a value greater than 0.01. The maximum is 100.\par
The effect will be weaker on farther pixels,\par
the change is represented by a Gamma curve.\par
When the value is 1.0 brightness will be linearly attenuated.\par
The smaller the value, the narrower the brightness becomes (the effects is reduced\par 
more abruptly),\par
The larger the value, the broader the brightness becomes (the effect is emphasized).\par
The default value is 1.

\newpage

\thispagestyle{empty}

\ \vspace{-0.2em}
\par
\noindent Power\par
Changes the intesity of the light.\par
Specify a value in the range 0 to 1.\par
It is possible to specify values above 1.0, up to a maximum 2.0, to emphasize the\par 
light.\par
Such an emphasis may prove useful to emit light from dark portions such as ink\par 
lines.\par
When the value is 0.0 light will not be scattered, and fog will not be applied.\par
It is possible to specify negative values, down to a minimum of -2.0.\par
In such a case there will be no light scattering, but darkness scattering.\par
The default value is 1.\\
\\
Threshold Min\\
Threshold Max\par
Pixels with this luminosity value or higher will emit light.\\
\par
In additionn to being affected by the brightness from brighter neraby pixels,\par
pixels with values higher than (\textquotedbl Threshold Min\textquotedbl ) will emit light themselves.\par
The brightness, will be determined by the L value (of HLS) of the pixels.\\
\par
Values specified can range from 0.0 to 1.01.\\
\par
If both values are set to 1.01, fog will not be applied.\\
\par
When \textquotedbl Threshold Max\textquotedbl \ is greater than \textquotedbl Threshold Min\textquotedbl ,\par
the fog intensity will change linearly from Min and Max.\\
\par
By setting \textquotedbl Threshold Max\textquotedbl \ to zero (and less than Min),\par
light will be emitted from pixels with a brightness of \textquotedbl Threshold Min\textquotedbl \ or more.\par
Setting \textquotedbl Threshold Min\textquotedbl \ to 0 will produce a full fog.\\
\par
The default value is 0 for both parameters.\\
\\
Alpha Rendering\par
This option is valid only when there is an Alpha channel.\par
When inactive, it masks the changes in the RGB values using the original Alpha\par 
of the image.\par
When active, the effect will be able to modify the Alpha channel, extending it\par 
as necessary to reproduce the full span of the effect.\par
The default setting is inactive.

\end{document}