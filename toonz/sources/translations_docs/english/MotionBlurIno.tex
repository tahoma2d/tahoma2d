\documentclass[a4paper,12pt]{article}
\usepackage[a4paper, total={180mm, 272mm}]{geometry}

\usepackage{fontspec}
\setmainfont[Path=fonts/, Extension=.ttf]{ipaexm}

\setlength\parindent{3.5em}
\setlength\parskip{0em}
\renewcommand{\baselinestretch}{1.247}

\begin{document}

\thispagestyle{empty}

\Large
\noindent \\
Motion Blur Ino\medskip
\par
\normalsize
Produces a linear motion blur effect by using translation values.\par
It is also possible to specify an afterimage shake effect in the options.\\
\par
When \textquotedbl Alpha Rendering\textquotedbl \ is ON, Alpha channel will be processed first,\par
then it will process all RGB pixels where the Alpha channel is not zero.\\
\par
When \textquotedbl Alpha Rendering\textquotedbl \ is OFF, Alpha channel will not be taken into account,\par
so the RGB image changes will not be masked, causing defined edges.\\
\\
-{-}- \ Inputs \ -{-}-\\
Source\par
Connect the image to be processed.\\
\\
-{-}- \ Settings \ -{-}-\\
Depend Move\par
\noindent \ \ \, P1 -> P2\par
Use the X1,Y1,X2,Y2 parameters to specify movement in a fixed direction and\par 
magnitude.\\
\par
\noindent \ \ \, Motion\par
It will take into account the object movements along the E/W and N/S channels,\par 
on a frame-by-frame basis.\par
The X1,Y1,X2,Y2 parameters values will be ignored.\\
\\
X1\\
Y1\\
X2\\
Y2\par
Specifies the start and end coordinate values for the motion blur.\par
The origin of the used coordinate system is at the lower left corner.\par
The unit is millimeters.\par
By specifying values with decimal places, it will be possible to define subtle changes\par 
in the length.\par
If the distance between the starting and ending points is less than 1/16 of a pixel,\par 
it will have no effect.\par
The default values are\par
\noindent \hskip 7em X1 Y1 -> 0.0 0.0\par
\noindent \hskip 7em X2 Y2 -> 1.0 1.0\par

\newpage

\thispagestyle{empty}

\ \vspace{-0.2em}
\par
\noindent Scale\par
Adjusts the scale for the length of the motion blur effect.\par
For example,\par
\noindent \hskip 7em X1 Y1 -> 0.0 0.0\par
\noindent \hskip 7em X2 Y2 -> 1.0 -1.0\par
\noindent \hskip 7em Scale -> 100\par
Would be equivalent to,\par
\noindent \hskip 7em X1 Y1 -> 0.0 0.0\par
\noindent \hskip 7em X2 Y2 -> 100.0 -100.0\par
and will have the same effect.\par
When Scale is 0, blur will not be applied.\par
The default value is 1, which will apply no scaling.\\
\\
Curve\par
Allows to adjust the blur strenght.\par
With a value between 1.0 and 10.0, blur will become stronger.\par
With a value between 0.1 and 1.0, blur will become lighter.\par
The default value is 1, which will produce an linear attenuation.\\
\\
Afterimage Length\par
Specify the lenght of the afterimage effect.\par
The unit is millimeters.\par
Specify a value greater than or equal to 0.\par
For example, to create an afterimage for a line with a width of 3,\par
specify a value greater than or equal to 3.\par
The default value is 0, which produces no afterimage.\\
\\
Afterimage Power\par
Determines the strength of the afterimage.\par
At 0, no afterimage effect will be applied.\par
The larger the value, the less blur and stronger afterimage effect will be applied.\par
The default value of 1 is the strongest possible one. Where there will be no\par 
motion blur, but only the afterimage effect applied.\\
\\
Alpha Rendering\par
This option is valid only when there is an Alpha channel.\par
When inactive, it masks the changes in the RGB values using the original Alpha\par 
of the image.\par
When active, the effect will be able to modify the Alpha channel, extending it\par 
as necessary to reproduce the full span of the effect.\par
The default setting is active.

\end{document}