\documentclass[a4paper,12pt]{article}
\usepackage[a4paper, total={180mm, 272mm}]{geometry}

\usepackage{fontspec}
\setmainfont[Path=fonts/, Extension=.ttf]{ipaexm}

\setlength\parindent{3.5em}
\setlength\parskip{0em}
\renewcommand{\baselinestretch}{1.247}

\begin{document}

\thispagestyle{empty}

\Large
\noindent \\
Negate Ino\medskip
\par
\normalsize
Inverts each color channel in the image.\\
\par
\ \ 8bits image: 0-{-}> \ \ \ 255、1-{-}> \ \ \, 254、... \ \ \ \ 254-{-}>1、 \ \ \, 255-{-}>0\par
16bits image: 0-{-}>65535、1-{-}>65534、... 65534-{-}>1、65535-{-}>0\par
Values are inverted as in the above example.\\
\par
When there is an Alpha channel, the inversion takes into account the Alpha values.\par
Alpha Value \ \ \ \ 0: 0-{-}> \ \ \, 0、1-{-}> \ \ \ 0、... 254-{-}>0、255-{-}>0\par
Alpha Value 128: 0-{-}>128、1-{-}>127、... 127-{-}>1、128-{-}>0、...255-{-}>0\par
Alpha Value 255: 0-{-}>255、1-{-}>254、... 254-{-}>1、255-{-}>0\par
Subsequently, cell images will be reversed as expected.\\
\\
-{-}- \ Inputs \ -{-}-\\
Source\par
Connect the image to be processed.\\
\\
-{-}- \ Settings \ -{-}-\\
Allows to specify whether to do the inversion in each color channel.\\
\\
Red \ \ \ \, Red channel inversion switch\\
Green \ Green channel inversion switch\\
Blue \ \ \, Blue channel inversion switch\\
Alpha \ \ Alpha channel inversion switch\\
\\
When inactive it does not do anything to that channel.\\
When active, it will invert the target channel.\\
The default settings for Red, Green, Blue is active, and Alpha is inactive.

\end{document}